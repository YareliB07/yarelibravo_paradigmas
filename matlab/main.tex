\documentclass[12pt]{article}
\usepackage[utf8]{inputenc}
\usepackage{parskip}
\usepackage{graphicx}
\usepackage{subfig}
\usepackage[left= 2.5cm, right=2.5cm]{geometry}

\title{\includegraphics[scale = 0.1]{LogoUv.png}}

\author{
        Universidad Veracruzana \\
        Facultad de Negocios y Tecnologías\\
        Orizaba --- Córdoba\\
        Ingeniería de Software\\
        \textbf{Reporte Técnico. Red Neuronal}\\
        Paradigmas de Programación\\
        \textbf{Presenta}\\
        Yareli Sugey Bravo Morales\\
        \textbf{Maestro}\\
        Adolfo Centeno Tellez\\
        17 de marzo de 2022\\
}

\date{}

\begin{document}

\maketitle

\newpage


\section{Introducción}

Con el objetivo de ampliar nuestro conocimiento dentro de la experiencia educativa de “paradigmas de programación”, se propuso realizar una práctica acerca del reconocimiento de patrones dentro de una red neuronal artificial. Con el fin de tener una idea más clara de este tema, se decidió realizar un reporte técnico en el cual se proporcionará una pequeña definición de las Redes Neuronales Artificiales (RNA) y la Red Hopfield, de igual modo, se describirá brevemente el funcionamiento del código que fue implementado dentro de esta práctica. 

\section{Desarrollo}
\section*{¿Qué es una Red Neuronal?}

Las redes neuronales artificiales son un modelo inspirado en el funcionamiento del cerebro humano. Está formado por un conjunto de nodos conocidos como neuronas artificiales que están conectadas y transmiten señales entre sí. Estas señales se transmiten desde la entrada hasta generar una salida. Las RNA utilizan grafos y funciones, conformadas por elementos de proceso (EP o nodos) y conexiones (enlaces). Procesan entradas y generan salidas que ayudan a resolver problemas. 

Hay tres partes normalmente en una red neuronal: 
\begin{itemize}
    \item Capa de entrada. Neuronas que reciben datos o señales procedentes del entorno.
    \item Capa oculta. No tiene conexión directa con el entorno y están fuertemente conectados entre sí. 
    \item Capa de salida. Neuronas que proporcionan la solución de la RN.

\end{itemize} 


\begin{center}
    \includegraphics[scale = 0.4]{Red_Neuronal.jpeg}
\end{center}

Las RNA tienen una función de activación que calcula el estado de actividad de una neurona; transformando la entrada global (menos el umbral) en un valor (estado) de activación, cuyo rango normalmente va de (0 a 1) o de (–1 a 1). Esto es así, porque una neurona puede estar totalmente inactiva (0 o –1) o activa (1).

Una red neuronal debe aprender a calcular la salida correcta para cada arreglo o vector de entrada, por eso, una vez diseñada la arquitectura de la red y las funciones que la regirán, se tiene que entrenar a la red para que aprenda el comportamiento que debe tener, a este proceso de aprendizaje se le conoce como proceso de entrenamiento o acondicionamiento. 

Una excepción a esta regla general son las Redes de Hopfield, que no son entrenadas sino construidas, de modo que tengan inicialmente el comportamiento deseado. 

\section*{Ventajas de una Red Neuronal}

\begin{itemize}
    \item Aprendizaje Adaptativo. Capacidad de aprender a realizar tareas basadas en un entrenamiento o en una experiencia inicial.
    \item Auto-organización. Una red neuronal puede crear su propia organización o representación de la información que recibe mediante una etapa de aprendizaje. 
    \item Tolerancia a fallos. La destrucción parcial de una red conduce a una degradación de su estructura; sin embargo, algunas capacidades de la red se pueden retener, incluso sufriendo un gran daño.
\end{itemize} 

\section*{¿Qué es una Red Hopfield?}

La Red de Hopfield es una red recurrente, es decir, existe realimentación entre las neuronas. De esta forma, al introducir un patrón de entrada, la información se propaga hacia adelante y hacia atrás, produciéndose una dinámica. En algún momento, la evolución se detendrá en algún estado estable. 

\section*{Arquitectura}

El modelo de la red Hopfield es una red monocapa con N neuronas cuyas salidas toman el valor de 0/1. Cada neurona se conecta a las demás, apareciendo conexiones laterales, pero no se conectan consigo mismas por lo que no existen conexiones auto-recurrentes, además, los pesos asociados a pares de neuronas son simétricos, es decir, se da la igualdad Wij=Wji. El tipo de neurona usado en el trabajo original de Hopfield es la neurona de McCulloch-Pitts. 

    \begin{center}
    \includegraphics[scale = 0.4]{red_hopfield2.jpeg}
    \end{center}

0i es el umbral de disparo de la neurona i, que representa el desplazamiento de la función de transferencia a lo largo del eje de ordenadas (x). En este modelo suele adoptarse un valor proporcional a la suma de los pesos de las conexiones de cada neurona con el resto: 
    
    \begin{center}
    \includegraphics[scale = 0.4]{aaaa.jpeg}
    \end{center}
    
    \section*{Funcionamiento}
    
El primer paso a la hora de trabajar con una red de Hopfield es codificar y representar la información en forma de vector. Esta codificación será binaria si se usa la neurona McCulloch-Pitts o continua si se trata de una neurona no lineal que utiliza como función de activación la sigmoide (0 y 1) o la tangente hiperbólica (-1 +1). 

El vector de entrada tendrá tantas componentes como neuronas tenga la red. La entrada es aplicada en t = 0 a la única capa que tiene la red, determinándose así las salidas vj(0). Debido a las realimentaciones, estas salidas se convierten en las nuevas entras a la red.

Centrándonos en una sola neurona el funcionamiento sería el siguiente:

\begin{enumerate}
    \item Recibe como entrada la salida de cada una de las otras neuronas (por las conexiones laterales). Estos valores de salida inicialmente coinciden con las entradas del vector, multiplicadas por los pesos de las conexiones correspondientes. La suma de todos estos valores constituirá el valor de entrada neta de la neurona a la que hay que aplicarle la función de transferencia obteniéndose el valor de salida correspondiente.
    
    \begin{center}
    \includegraphics[scale = 0.6]{unooo.jpeg}
    \end{center}
    
    \item Este proceso continúa hasta que las salidas de las neuronas se estabilizan, alcanzan la convergencia, durante algunas iteraciones.
    
    \begin{center}
    \includegraphics[scale = 0.6]{dosss.jpeg}
    \end{center}
    
\end{enumerate}


\section*{Aprendizaje de una Red Hopfield}
El mecanismo de aprendizaje utilizado es de tipo OFF LINE, por lo que existe una etapa de aprendizaje y otra de funcionamiento de la red. También utiliza un aprendizaje no supervisado de tipo hebbiano, de tal forma que el peso de una conexión entre una neurona i y otra j se obtiene mediante el producto de los componentes i-ésimo y j-ésimo del vector que representa la información o patrón que debe almacenar.

Utilizando una notación matricial, para representar los pesos de la red se puede utilizar una matriz de dimensión NxN (recordemos que N es el número de neuronas de la red y por tanto de componentes del vector de entrada). Esta matriz es simétrica (wij = wji) y con la diagonal con valores nulos (wii = 0) al no haber conexiones autorecurrentes.


\begin{center}
W = Sumk=1..N (T(Ek). Ek - I)
\end{center} 

\section*{\textbf Representación del Modelo Básico de Hopfield}

\begin{center}
    \includegraphics[scale = 0.3]{Red_Hopfield.jpeg}
\end{center}

\begin{itemize}
    \item n es el número de nodos en la red. 
    \item Las entradas Xo , X1 ... Xn-1 se aplican a la red en el tiempo t = 0. Pueden tomar valores de +1 ó -1.
    \item Las salidas Uo , U1 ... Un-1 se van calculando y recalculando, hasta que sus valores ya no cambian. Cuando esto sucede, se tiene la salida de la red, y X’i = Ui para i= 1.. n-1
\end{itemize}

\section*{Ventajas y desventajas}

\begin{itemize}
   
    \item El número de patrones a almacenar es bastante limitado comparado con el número de nodos en la red. Según Hopfield, el número de clases a aprender no puede ser mayor de 0.15 veces el número de nodos en la red.
    \item Las redes de Hopfield son bastante tolerantes al ruido, cuando funcionan como memorias asociativas.
    
\end{itemize}

\section*{Práctica Red Neuronal}

Una vez explicada de manera breve la teoría, podemos adentrarnos a la parte práctica de este reporte. Para este proyecto, construí 5 matrices de 12 x 12 con diferentes patrones de figuras geométricas (cuadrado, triángulo-rectángulo, rectángulo, estrella y corazón).  
      
         \begin{center}
            \includegraphics[width=0.4\textwidth]{cuadrado.jpeg}
            \includegraphics[width=0.4\textwidth]{estrella.jpeg} 
            \includegraphics[width=0.4\textwidth]{trianguloR.jpeg}
            \includegraphics[width=0.4\textwidth]{corazon.jpeg} 
            \includegraphics[width=0.4\textwidth]{rectangulo.jpeg}
         \end{center}

Al ser una matriz de 12 x 12, mi Red Hopfield contará con 144 neuras, procedí a colocar en un solo renglón y cambiando los 0 de mi matriz a -1 y los 1 a 1. 

cuadrado= [-1 -1 -1 -1 -1 -1 -1 -1 -1 -1 -1 -1 -1 -1 -1 -1 -1 -1 -1 -1 -1 -1 -1 -1 -1 -1 1 1 1 1 1 1 1 1 -1 -1 -1 -1 1 -1 -1 -1 -1 -1 -1 1 -1 -1 -1 -1 1 -1 -1 -1 -1 -1 -1 1 -1 -1 -1 -1 1 -1 -1 -1 -1 -1 -1 1 -1 -1 -1 -1 1 -1 -1 -1 -1 -1 -1 1 -1 -1 -1 -1 1 -1 -1 -1 -1 -1 -1 1 -1 -1 -1 -1 1 -1 -1 -1 -1 -1 -1 1 -1 -1 -1 -1 1 1 1 1 1 1 1 1 -1 -1 -1 -1 -1 -1 -1 -1 -1 -1 -1 -1 -1 -1 -1 -1 -1 -1 -1 -1 -1 -1 -1 -1 -1 -1];
           
estrella= [-1 -1 1 -1 -1 -1 -1 -1 -1 1 -1 -1 -1 -1 1 1 1 -1 -1 1 1 1 -1 -1 -1 -1 1 -1 -1 1 1 -1 -1 1 -1 -1 -1 -1 -1 1 -1 -1 -1 -1 1 -1 -1 -1 -1 -1 -1 1 -1 -1 -1 -1 1 -1 -1 -1 -1 -1 1 -1 -1 -1 -1 -1 -1 1 -1 -1 -1 1 -1 -1 -1 -1 -1 -1 -1 -1 1 -1 1 1 1 1 1 -1 -1 1 1 1 1 1 -1 -1 -1 -1 -1 1 1 -1 -1 -1 -1 -1 -1 -1 -1 -1 -1 1 1 -1 -1 -1 -1 -1 -1 -1 -1 -1 -1 1 1 -1 -1 -1 -1 -1 -1 -1 -1 -1 -1 1 1 -1 -1 -1 -1 -1];
                     
rectangulo= [-1 1 1 1 1 1 1 1 1 1 1 -1 -1 1 -1 -1 -1 -1 -1 -1 -1 -1 1 -1 -1 1 -1 -1 -1 -1 -1 -1 -1 -1 1 -1 -1 1 -1 -1 -1 -1 -1 -1 -1 -1 1 -1 -1 1 -1 -1 -1 -1 -1 -1 -1 -1 1 -1 -1 1 -1 -1 -1 -1 -1 -1 -1 -1 1 -1 -1 1 -1 -1 -1 -1 -1 -1 -1 -1 1 -1 -1 1 -1 -1 -1 -1 -1 -1 -1 -1 1 -1 -1 1 -1 -1 -1 -1 -1 -1 -1 -1 1 -1 -1 1 -1 -1 -1 -1 -1 -1 -1 -1 1 -1 -1 1 -1 -1 -1 -1 -1 -1 -1 -1 1 -1 -1 1 1 1 1 1 1 1 1 1 1 -1];
  
corazon= [-1 -1 -1 -1 -1 -1 -1 -1 -1 -1 -1 -1 -1 1 1 1 -1 -1 -1 -1 1 1 1 -1 1 1 1 1 1 -1 -1 1 1 1 1 1 1 1 1 -1 1 1 1 1 -1 -1 1 1 1 1 1 -1 -1 1 1 -1 -1 -1 1 1 1 1 1 -1 -1 -1 -1 -1 -1 -1 1 1 -1 1 1 -1 -1 -1 -1 -1 -1 1 1 -1 -1 -1 1 1 -1 -1 -1 -1 1 1 -1 -1 -1 -1 -1 1 1 -1 -1 1 1 -1 -1 -1 -1 -1 -1 -1 1 1 1 1 -1 -1 -1 -1 -1 -1 -1 -1 -1 1 1 -1 -1 -1 -1 -1 -1 -1 -1 -1 -1 -1 -1 -1 -1 -1 -1 -1];
                     
triangulorec= [1 -1 -1 -1 -1 -1 -1 -1 -1 -1 -1 -1 1 1 -1 -1 -1 -1 -1 -1 -1 -1 -1 -1 1 -1 1 -1 -1 -1 -1 -1 -1 -1 -1 -1 1 -1 -1 1 -1 -1 -1 -1 -1 -1 -1 -1 1 -1 -1 -1 1 -1 -1 -1 -1 -1 -1 -1 1 -1 -1 -1 -1 1 -1 -1 -1 -1 -1 -1 1 -1 -1 -1 -1 -1 1 -1 -1 -1 -1 -1 1 -1 -1 -1 -1 -1 -1 1 -1 -1 -1 -1 1 -1 -1 -1 -1 -1 -1 -1 1 -1 -1 -1 1 -1 -1 -1 -1 -1 -1 -1 -1 1 -1 -1 1 -1 -1 -1 -1 -1 -1 -1 -1 -1 1 -1 1 1 1 1 1 1 1 1 1 1 1 1];
               

\section{Conclusión}
Las Redes Neuronales Artificiales son un intento de imitar nuestra forma de pensar, estas se han abierto gran paso en la sociedad, actualmente, son muy importantes ya que son parte de las bases de la inteligencia artificial, sin darnos cuenta, hemos visto alguna red, como en el reconocimiento de patrones, identificación facial, etc. 

\section{Bibliografía}
\begin{itemize}
    \item del Pilar, M. (2017). Las Redes de Hopfield [Diapositivas]. PowerPoint. 
    
    https://ccc.inaoep.mx/~pgomez/cursos/redes-20neuronales-20artificiales/presentaciones
    
    /hopfield.pdf
    
    \item El modelo de redes neuronales. (s. f.). IBM. Recuperado 17 de marzo de 2022, de https://www.ibm.com/docs/es/spss-modeler/SaaS?topic=networks-neural-
    modelSerrano, A. (2010). Redes 
\end{itemize}
\end{document}
